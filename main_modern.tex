\documentclass[10pt,a4paper,sans]{moderncv2}

\moderncvstyle{classic}
\moderncvcolor{blue}

%\usepackage[maxnames=10, backend=bibtex8]{biblatex}
\usepackage[utf8]{inputenc}
\usepackage[scale=0.8]{geometry}

\input{personaldata.tex} %Use personaldata_d.tex as template
%\bibliography{publications}

\makeatletter
\name{Lorenzo}{Ferrillo}
\title{Curriculum Vitae}
\email{\@@@email}
\phone[mobile]{\@@@phone}
\phone[fixed]{\@@@homeph}
\extrainfo{Date of birth: 07/01/1995}
\social[github][github.com/llde]{github.com/llde}
\social[linkedin][www.linkedin.com/in/lorenzo-ferrillo-328b28156]{lorenzo-ferrillo-328b28156/}
\photo{personal.jpg}
\makeatother
%\makeatletter\renewcommand*{\bibliographyitemlabel}{\@biblabel{\arabic{enumiv}}}\makeatother

\def\pplus{\texttt{++}}
\def\sharp{\texttt{\#}}

%\renewcommand{\refname}{Research activities and publications}
%\DefineBibliographyExtras{english}{\let\finalandcomma=\empty}
%\newcommand{\pubitem}[1]{
%    \cvitem{\citefield{#1}{year}}{\textbf{\citefield{#1}{title}} (with \citename{#1}{author}), in \textit{\citefield{#1}{booktitle} } }
%}

\begin{document}
    \makecvtitle

    \section{Education}
    \cventry{2022}{Bachelor Degree in Computer Science}{Sapienza University of Rome, Italy}{}{}{
        Final mark: \emph{100/110}\\
        Thesis title: \emph{Rilevamento di vulnerabilità tramite reti neurali a convoluzione}.\\
        Advisors: \emph{Prof. Fabio De Gaspari}.\\
        Description: \emph{We present a system for vulnerability detection using convolutional neural network and scripts to prepare the dataset for training}.
    }

    \section{Computer skills}
    \cvitem{Programming}{
        Fluent in Rust, Python and C.
    }
    \cvitem{}{
        Good knowledge of C\pplus{}, Java  and Intel x86 Assembly programming languages.
    }
    \cvitem{}{
        Knowledge of SQL,  \LaTeX, C\sharp{} and HLSL.
        Basic knowledge in HTML and CSS.
    }
    \cvitem{Development tools}{
        Good knowledge of Git and the Git workflow. 
        My main development tools include JetBrains IDEs (CLion, IntelliJ Idea, PyCharm), Microsoft Visual Studio and Visual Studio Code, Powershell, zsh, Notepad\pplus{} and Kate.
    }
    \cvitem{Reverse Engineering}{
        Medium knowledge for Disassemblers and decompilers, as IDA Pro and Cutter (Rizin/Radare2)
    }
    \cvitem{Operating systems}{
        Expert in GNU/Linux, in particular Ubuntu Fedora and Arch Linux distributions. \newline
        Good knowledge of Windows operating systems, in particular Windows 7/8/10. Member of the Windows Insider program. \newline
        Limited knowledge on other Unix OSes, as OSX, WebOS and BSD systems.
    }
    
    \section{University projects}
    \cvitem{Crawly}{A Concurrent Web Crawler made in Java using the JavaFX library (for "Metodology of Programming"). }
    \cvitem{Crypto-Sithis}{A Client/Server program using pseudo-random numbers to crypt files on disk (for "System Programming").}
    \cvitem{311-X}{Software analysis and architecture engineering for a multi-application Uber-like service (for "Software Engineering"). }
    \cvitem{City-builder}{Procedural generation of the buildings of a city, using the yocto-gl library for rendering (for "Computer Graphic)."}
    \section{Personal projects}
    \cvitem{xOBSE}{An injector to extend the engine capabilities for the game "The Elder Scroll 4: Oblivion", using C++ and inline Microsoft Assembly (MASM).}
    \cvitem{TESReloaded}{An extender for TES4: Oblivion, TES5: Skyrim (legacy/oldrim) and Fallout: New Vegas to introduce new graphical capabilities to the engine. }
    \cvitem{hlslDecompiler}{Decompiler for translation of the DirectX shader bytecode to hlsl (Work in progress). }
    \section{Open Source Contributions}
    \cvitem{}{
         Contributor in the following projects: Ureq, roctogen, patch-rs, Wine, Wine-Mono, Wine-Tkg, Wine-Staging and Wrye-Bash 
    }
    
    \section{Languages}
    \cvitem{Italian}{Mother tongue.}
    \cvitem{English}{Self-assessment (Common European Framework of Reference for Languages, CEFR).}
    \cvitem{}{\begin{tabular}{l@{\hspace{.5cm}}l}
        Reading: & C1 (Proficient user)\\
        Listening: & B2 (Indipendent user)\\
        Writing: & C1 (Proficient user)\\
        Speaking: & B1 (Indipendent user)
    \end{tabular}}

    \section{Spare time activities}
    \cvitem{Hobbies}{
        Videogames: mainly strategic and RPG videogames. \newline 
        Games: Chess, Reversi, Card games   \newline
        Music: Symphonic Metal and Power Metal listener. \newline
        Open air activities: Trekking, Swimming. \newline
        Sports: Basketball, Tennis Table 
    }

    \cvitem{Interests}{Programming, Modding, History, Science, general culture.}

\begin{footnotesize}
\vspace{14cm}
Autorizzo il trattamento dei miei dati personali presenti nel cv ai sensi del Decreto Legislativo 30 giugno 2003, n. 196 “Codice in materia di protezione dei dati personali” e dell’art. 13 del GDPR                 (Regolamento UE 2016/679).
\end{footnotesize}  
\end{document}